\documentclass[10pt,twocolumn,letterpaper]{article}

\usepackage{cvpr}
\usepackage{times}
\usepackage{epsfig}
\usepackage{graphicx}
\usepackage{amsmath}
\usepackage{amssymb}
\usepackage{placeins}
\usepackage{csvsimple, booktabs}
% \csvset{latexfriendly=true}
% Include other packages here, before hyperref.

% If you comment hyperref and then uncomment it, you should delete
% egpaper.aux before re-running latex.  (Or just hit 'q' on the first latex
% run, let it finish, and you should be clear).
\usepackage[pagebackref=true,breaklinks=true,letterpaper=true,colorlinks,bookmarks=false]{hyperref}

\cvprfinalcopy % *** Uncomment this line for the final submission

\def\cvprPaperID{****} % *** Enter the CVPR Paper ID here
\def\httilde{\mbox{\tt\raisebox{-.5ex}{\symbol{126}}}}

% Pages are numbered in submission mode, and unnumbered in camera-ready
\ifcvprfinal\pagestyle{empty}\fi
\begin{document}

%%%%%%%%% TITLE
\title{[CS 7643] Classifying Useful Tweets during Disaster}

\author{Jinwoo Jeong\\
{\tt\small jjeong342@gatech.edu}
% For a paper whose authors are all at the same institution,
% omit the following lines up until the closing ``}''.
% Additional authors and addresses can be added with ``\and'',
% just like the second author.
% To save space, use either the email address or home page, not both
\and
Jade Kim\\
{\tt\small jkim4163@gatech.edu}
}

\maketitle
%\thispagestyle{empty}

%%%%%%%%% ABSTRACT
\begin{abstract}
   The ABSTRACT is to be in fully-justified italicized text, at the top
   of the left-hand column, below the author and affiliation
   information. Use the word ``Abstract'' as the title, in 12-point
   Times, boldface type, centered relative to the column, initially
   capitalized. The abstract is to be in 10-point, single-spaced type.
   Leave two blank lines after the Abstract, then begin the main text.
   Look at previous CVPR abstracts to get a feel for style and length. 
   The abstract section should contain a brief summary of your work that
   includes the problem statement, proposed solution and results.
\end{abstract}

%%%%%%%%% BODY TEXT
\section{Introduction} % \section{Introduction/Background/Motivation}

% \textit{(5 points) \textbf{What did you try to do? What problem did you try to solve?} Articulate your objectives using absolutely no jargon.}

In the event of a disaster, valuable real-time information can be found on social media platforms such as Twitter (now known as X) \cite{TwitterInCrisis}. Time-critical information, such as reports of injured or trapped people, is essential for organizations seeking to minimize human casualties. Information on the need for food, water, or medical support is also highly valuable for both organizations and volunteers assisting survivors. However, it is challenging and time-consuming to identify such useful information within the overwhelming stream of online messages, as it is often mixed with irrelevant or non-actionable content \cite{BigCrisisData}.

% \textit{(5 points) How is it done today, and what are the limits of current practice?}

To address this issue, researchers have increasingly explored how social media data can be analyzed during crisis. A large body of work has especially focused on Twitter data for purposes such as early disaster detection, situational awareness, and extracting useful real-time information \cite{CrisisBenchPaper, TwitterInCrisis}. In particular, a substantial portion of research has focused on classification tasks, supported by the increasing availability of labeled Twitter datasets (e.g., CrisisBench \cite{CrisisBenchPaper}, CrisisLex \cite{Crisislex}, CrisisNLP \cite{CrisisNLP}, CrisisMMD \cite{CrisisMMD}). 

Twitter-based classification studies span a wide range of objectives. In informativeness classification, the goal is to determines whether a tweet contains actionable or useful information during a crisis. This binary classification (informative vs. non-informative) is useful for filtering out tweets that lack practical value. The humanitarian classification task, in contrast, categorizes tweets into specific humanitarian information types (e.g., reports of injured or missing individuals, infrastructure damage), which is especially critical for effective decision-making during crisis situations. Such fine-grained categorization enables responders to understand what type of support is needed, where it is needed, and for whom \cite{CrisisBenchPaper}.
 
% A separate line of work classifies tweets by event type (e.g., floods, earthquakes, wildfires)

% [non-academic => remove] Information gathering during disasters often relies on simple keyword and hashtag searching, but this approach is prone to failure due to many reasons. For instance, disaster-related terminology is often diverse and ambiguous (e.g., “earthquake”, “shaking”, “payers”) and some messages even use incorrect terminology \cite{Wiegmann2020}.

% \textit{(5 points) What did you try to do? What problem did you try to solve? \textbf{Articulate your objectives using absolutely no jargon.}}

However, both of these classification tasks have limitations. Binary classification produces informative tweets, but they must be further classified for practical use. Humanitarian classification task produces labels that is more useful, but the effectiveness of this task varies based on the number and composition of classes. For instance, the humanitarian task dataset in CrisisBench consists of 11 target labels, and classifying tweets into too many detailed categories can introduce complexity and ambiguity. For instance, a tweet that reports that someone's injured and under a collapsing building should be classified as both injured people or infrastructure damage categories, but it can only fall under one of them, making practical deployment difficult. 

Thus, we have defined a multi-class classification task with three categories based on how each tweet can be utilized during disasters. Tweets that convey information needed for handling urgent incidents requiring immediate action (e.g., reports of injured people or infrastructure damage) should be classified as the time-critical class. Tweets containing information useful for assisting rescued or evacuated survivors are labeled as the support-and-relief class. Tweets that do not contain any actionable or relevant information (e.g., random online users' reactions to the disaster) fall under the non-informative class.

To solve this task, multiple deep learning models with different architectures were trained to classify tweets into these three defined categories. We will focus on convolutional neural networks (CNN) and transformer architecture as they have been used in verious crisis datasets and have shown high performance \cite{CrisisBenchPaper, Wiegmann2020, Ning}.

% \textit{(5 points) Who cares? If you are successful, what difference will it make?}
If we can successfully train such classification system, formal organizations and volunteers will be able to efficiently extract useful information from flood of online tweets during crisis, which will be useful for allocating resources. Non-informative tweets could be discarded to save time. On-site rescue teams and 119 operators will greatly benefit from information conveyed in tweets classified as time-critical. Tweets classified as support-and-relief will be useful for organizing donations for the survivors who need food, water, or medical support.

\subsection{Dataset}
%(5 points) What data did you use? Provide details about your data, specifically choose the most important aspects of your data mentioned \href{https://arxiv.org/abs/1803.09010}{here}. You don’t have to choose all of them, just the most relevant.

This study uses the \textbf{CrisisBench} dataset~\cite{CrisisBenchPaper} which serves as a major benchmark for crisis classification research. As detailed in the original paper, the dataset comprises tweets collected from various crisis situations and classifies the information into 11 categories based on humanitarian needs. The data exhibits characteristics typical of social media text: it is short, informal, and noisy. It also contains misspellings, emojis, and non-standard language. Following the guidelines from \emph{Datasheets for Datasets}~\cite{DatasheetsPaper}, we document key aspects of the dataset (motivation, composition, collection process, recommended uses).
\subsubsection{Dataset Size}
The CrisisBench dataset was constructed by consolidating eight pre-existing crisis-related Twitter datasets. The initial merge resulted in a total of \textbf{206,411 tweets} before any mapping or filtering was applied. After performing class-label harmonization and removing duplicates, the authors obtained:
\begin{itemize}
    \item \textbf{166,098 tweets} for the \textbf{informativeness} classification task, and
    \item \textbf{141,533 tweets} for the \textbf{humanitarian} classification task.
\end{itemize}
For the main experiments conducted in the original study which used only English-language tweets, the dataset size was further reduced to:
\begin{itemize}
    \item \textbf{156,899 English tweets} for informativeness, and
    \item \textbf{87,557 English tweets} for humanitarian categories.
\end{itemize}

\subsubsection{Labeling Process}
CrisisBench was created by integrating eight human-annotated source datasets, each containing its own labeling conventions. To construct a unified benchmark, the authors employed a manual label mapping and validation process. Key aspects of this procedure include:
\begin{enumerate}
    \item \textbf{Manual mapping of inconsistent labels:} Domain experts mapped semantically similar labels from different datasets into unified label clusters, ensuring consistent annotation across sources.
    \item \textbf{Focus on two core tasks:} The harmonization was designed to support classification of (1) \emph{informativeness} (informative vs.\ not-informative) and (2) \emph{humanitarian information types} (e.g., affected individuals, infrastructure damage).
    \item \textbf{Exclusion of non-humanitarian categories:} Labels irrelevant to crisis-response use cases were removed (e.g., ``animal management,'' ``not labeled''), reducing the number of usable tweets.
\end{enumerate}

As an example of the mapping process, the label \textit{``building damaged''} from the AIDR system was merged into the broader category \textit{``infrastructure and utilities damage''} in the final humanitarian taxonomy.

%------------------------------------------------------------------------
% -----------------------------------------------------------------------
\section{Approach}

(10 points) What did you do exactly? How did you solve the problem? Why did you think it would be successful? Is anything new in your approach? \\

(5 points) What problems did you anticipate? What problems did you encounter? Did the very first thing you tried work? \\

\textbf{Important: Mention any code repositories (with citations) or other sources that you used, and specifically what changes you made to them for your project. }

\subsection{Models}

In our preprocessing pipeline, we apply tokenization, lowercasing, and removal of URLs and user mentions (@username). We then restrict the dataset to the three selected humanitarian categories used in our study.

We evaluate three distinct model architectures:
\begin{enumerate}
    \item A \textbf{TextCNN} model utilizing pre-trained \textbf{GloVe} embeddings to assess its ability to extract local n-gram style features.
    \item A \textbf{custom transformer} model trained from scratch to capture global contextual relationships in the tweets.
    \item A \textbf{pretrained transformer} model that is fine-tuned on our classification task, representing a transfer learning approach leveraging large-scale language pretraining.
\end{enumerate}
By comparing these models, we aim to identify the architecture that offers the best performance for real-time disaster information filtering under our three-label formulation.


\subsection{Data Preprocessing}
Our preprocessing pipeline follows a structured set of steps designed to clean the CrisisBench dataset (\url{https://github.com/firojalam/crisis_datasets_benchmarks}) and prepare it for model training. The original dataset contains English-only humanitarian-labelled tweets in three splits (train, dev, test).
All preprocessing steps were implemented in a dedicated Jupyter notebook. The code for this project is available at \url{https://github.com/bugoverdose/dl-twitter-crisis}. \\
Preprocessing Pipeline:
\begin{itemize}
    \item Lower casing and removal of URLs, user mentions,symbols, emoticons, invisible and non-ASCII characters, punctuations (replaced with whitespace), numbers, and hashtag signs.
    \item Restriction to the three selected humanitarian categories used in our study.
\end{itemize}
\subsection{What's New in Our Approach?}
The main novelty in our approach is not in the individual cleaning steps, but in the re-framing of the CrisisBench classification task itself, moving from a 11-class problem to a triage-centric 3-class problem (time\_critical, support\_and\_relief, non\_informative), which prioritizes immediate operational utility for disaster response.
We anticipated success because:
\begin{itemize}
    \item Target Simplification: Reducing the 11 original labels to three relevant classes directly addresses the challenge of building an actionable, real-time triage system. This simplification aligns with the primary goal of surfacing only the most critical humanitarian information during disasters.
    \item Standardized Cleaning: The implemented text-cleaning steps (removal of URLs, hashtags, user mentions, and non-ASCII characters) follow established best practices for natural language processing on noisy social media data. These steps help ensure that the classification models focus on the core semantic content rather than those irrelevant to humanitarian decision-making.
\end{itemize}
\subsection{Problems}
\textbf{TODO: ALSO TALK ABOUT PROBLEMS FACED DURING EXPERIMENTATION SUCH AS COMPUTATION LIMITS} \\
We anticipated that the unstructured nature of social media text would lead to noise, and that much of the data would be irrelevant to the disaster response. Directly running the model on the 11 original labels would have likely yielded poor, non-actionable results due to the label complexity and imbalance, which is why the re-mapping was necessary before training. Our very first approach to preprocessing did not work because we failed to account for rows with empty content after cleaning. This led to errors during model training, and we later added logic to remove empty rows after the cleansing process.

\subsection{Performance}
We evaluated the performance of three distinct model architectures:
\begin{enumerate}
    \item A TextCNN model utilizing pre-trained GloVe embeddings to assess its ability to extract local n-gram style features.
    \item A custom transformer model trained from scratch to capture global contextual relationships in the tweets.
    \item A pretrained transformer model that is fine-tuned on our classification task, representing a transfer learning approach leveraging large-scale language pretraining.
\end{enumerate}
By comparing these models, we aim to identify the architecture that offers the best performance for real-time disaster information filtering under our three-label formulation.

\section{Experiments and Results}
\subsection{TextCNN}
\begin{figure}
    \centering
    \includegraphics[width=1.0  \linewidth]{cnn_flowchart.png}
    \caption{CNN Architecture}
    \label{fig:cnn_flowchart}
\end{figure}

\subsubsection{Architecture}
Our TextCNN model follows the architecture proposed in the original CrisisBench paper, adapted for our three-class classification task. TextCNN is particularly well suited for short, noisy text such as tweets because it captures local patterns such as short phrases indicative of urgent events through convolutional filters of different sizes. The model starts with an embedding layer initialized with pre-trained GloVe word embeddings (50-dimensional), followed by multiple parallel 1D convolutional layers with filter sizes of 3, 4, and 5 to capture n-gram features at different scales (refer to Figure~\ref{fig:cnn_flowchart}). Each convolutional layer uses 50 filters and applies ReLU activation, followed by max pooling over the sequence length. The outputs from all filter sizes are concatenated and passed through a dropout layer (dropout rate 0.5) before a fully connected layer that produces logits for the three classes (time\_critical, support\_and\_relief, non\_informative). The model uses a vocabulary size of 5000 and a maximum sequence length of 64 tokens.
\subsubsection{Hyperparameter Tuning}
We performed hyperparameter tuning to identify the optimal configuration for our TextCNN model. The hyperparameters that led to the best test F1 score (0.8237) are as follows: filter sizes of (3, 4, 5), 50 filters per filter size, dropout rate of 0.5, batch size of 32, learning rate of $1 \times 10^{-3}$, and training for 5 epochs. We used the Adam optimizer and evaluated our model based on the validation F1 score. We also performed a validation sweep over the number of filters (25, 35, 50, 60, 75, 85, 100). The validation curve (Figure~\ref{fig:cnn_curves}) shows that performance increases as the number of filters grows, peaking at 50 filters and declining afterward. This pattern indicates that too many filters lead to overfitting on local n-gram patterns, reducing generalization on noisy social media text.
\subsubsection{Learning and Validation Curves}
We plotted learning curves tracking both loss and macro F1 score across training epochs for both training and validation sets which is shown in Figure~\ref{fig:cnn_curves}. The curves show consistent improvement over 5 epochs, with the validation F1 score reaching 0.8247 at epoch 3. We also generated a validation curve by varying the number of filters while keeping other hyperparameters fixed, which helped us identify the optimal number of filters (50) for our model architecture. This observation reinforces that moderate-capacity CNNs are better suited for short-text classification, where excessively wide convolutional layers tend to memorize local token combinations rather than generalize.
Overall, the learning behavior demonstrates that TextCNN is efficient and capable of capturing the structure necessary for triage-focused disaster tweet classification.
\begin{figure}
    \centering
    \includegraphics[width=1.0\linewidth]{cnn_curves.png}
    \caption{CNN Learning and Validation Curves}
    \label{fig:cnn_curves}
\end{figure}
\subsubsection{Analysis of Results}

\subsection{Custom Transformer}
\begin{figure}
    \centering
    \includegraphics[width=1.0\linewidth]{custom_transformer_curves.png}
    \caption{Custom Transformer Learning and Validation Curves}
    \label{fig:custom_transformer_curves}
\end{figure}
\subsubsection{Architecture}

\subsubsection{Fine-tuning}

\FloatBarrier

\subsection{Fine-tuning Pretrained Transformer}

\subsubsection{Architecture}

\subsubsection{Fine-tuning}

\begin{figure}[h] % [t] 
\begin{center}
% \fbox{\rule{0pt}{2in} \rule{0.9\linewidth}{0pt}}
\includegraphics[width=1.05\linewidth]{pretrained_transformer_f1_score_curve.png}
\includegraphics[width=1\linewidth]{pretrained_transformer_f1_score_val_curve1.png}
\end{center}
   \caption{Example of caption.  It is set in Roman so that mathematics
   (always set in Roman: $B \sin A = A \sin B$) may be included without an
   ugly clash.}
\label{fig:long}
\label{fig:onecol}
\end{figure}


(10 points) How did you measure success? What experiments were used? What were the results, both quantitative and qualitative? Did you succeed? Did you fail? Why? Justify your reasons with arguments supported by evidence and data.

\textbf{Important: This section should be rigorous and thorough. Present detailed information about decision you made, why you made them, and any evidence/experimentation to back them up. This is especially true if you leveraged existing architectures, pre-trained models, and code (i.e. do not just show results of fine-tuning a pre-trained model without any analysis, claims/evidence, and conclusions, as that tends to not make a strong project). }

\FloatBarrier

%-------------------------------------------------------------------------
\section{Other Sections}

You are welcome to introduce additional sections or subsections, if required, to address the following questions in detail. 

(5 points) Appropriate use of figures / tables / visualizations. Are the ideas presented with appropriate illustration? Are the results presented clearly; are the important differences illustrated? 

(5 points) Overall clarity. Is the manuscript self-contained? Can a peer who has also taken Deep Learning understand all of the points addressed above? Is sufficient detail provided? 

(5 points) Finally, points will be distributed based on your understanding of how your project relates to Deep Learning. Here are some questions to think about: 

What was the structure of your problem? How did the structure of your model reflect the structure of your problem? 

What parts of your model had learned parameters (e.g., convolution layers) and what parts did not (e.g., post-processing classifier probabilities into decisions)? 

What representations of input and output did the neural network expect? How was the data pre/post-processed?
What was the loss function? 

Did the model overfit? How well did the approach generalize? 

What hyperparameters did the model have? How were they chosen? How did they affect performance? What optimizer was used? 

What Deep Learning framework did you use? 

What existing code or models did you start with and what did those starting points provide? 

Briefly discuss potential future work that the research community could focus on to make improvements in the direction of your project's topic.

\section{Work Division}

% Please add a section on the delegation of work among team members at the end of the report, in the form of a table and paragraph description. This and references do \textbf{NOT} count towards your page limit. An example has been provided in Table \ref{tab:contributions}.

Summary of contributions are provided in Table \ref{tab:contributions}.

{\small
\bibliographystyle{ieee_fullname}
\bibliography{egbib}
}

\begin{table*}
\begin{center}
\begin{tabular}{|l|p{8cm}|p{8cm}|} % {|l|c|p{8cm}|}
\hline
Student Name & Contributed Aspects & Details \\
\hline\hline
Jinwoo Jeong & Data Preprocessing, Transformer implementation, Training and Analysis & Preprocessed all the Tweeter data used in the project. Implemented custom transformer and fine-tuned pretrained transformer model. \\
Jade Kim & CNN Implementation, Training and Analysis & Implemented TextCNN model. Fine-tuned TextCNN and custom transformer. Analyzed effect of number of filters in TextCNN. \\
\hline
\end{tabular}
\end{center}
\caption{Contributions of team members.}
\label{tab:contributions}
\end{table*}

\FloatBarrier

\newpage



\appendix


% \section{CNN Hyperparameter Tuning Data}
% \begin{table*}[t]
% \centering
%     \begin{tabular}{|l|l|l|l|l|l|l|l|}
%     \hline
%         filters & filter\_sizes & dropout & batch\_size & lr & train\_f1 & val\_f1 & test\_f1 \\ \hline
%         50 & 3,4,5 & 0.5000 & 32 & 5e-4 & 0.8084 & 0.7950 & 0.7905 \\ \hline
%         50 & 3,4,5 & 0.5000 & 32 & 5e-4 & 0.8701 & 0.8246 & 0.8232 \\ \hline
%         50 & 3,4,5 & 0.7000 & 32 & 5e-4 & 0.8507 & 0.8237 & 0.8217 \\ \hline
%         50 & 4,5,6 & 0.7000 & 32 & 5e-4 & 0.8589 & 0.8214 & 0.8193 \\ \hline
%         100 & 3,4,5 & 0.7000 & 32 & 5e-4 & 0.8666 & 0.8216 & 0.8243 \\ \hline
%         100 & 4,5,6 & 0.7000 & 32 & 5e-4 & 0.8711 & 0.8285 & 0.8253 \\ \hline
%         50 & 3,4,5 & 0.7000 & 32 & 1e-3 & 0.8714 & 0.8258 & 0.8216 \\ \hline
%         50 & 4,5,6 & 0.7000 & 32 & 1e-3 & 0.8735 & 0.8246 & 0.8197 \\ \hline
%         100 & 3,4,5 & 0.5000 & 32 & 1e-3 & 0.8597 & 0.8253 & 0.8234 \\ \hline
%     \end{tabular}
%     \caption{Snippet of hyperparameter tuning values}
% \end{table}

\newpage
\newpage
\section{Miscellaneous Information}

The rest of the information in this format template has been adapted from CVPR 2020 and provides guidelines on the lower-level specifications regarding the paper's format.

\subsection{Language}

All manuscripts must be in English.


\subsection{Paper length}
Papers, excluding the references section,
must be no longer than six pages in length. The references section
will not be included in the page count, and there is no limit on the
length of the references section. For example, a paper of six pages
with two pages of references would have a total length of 8 pages.

%-------------------------------------------------------------------------
\subsection{The ruler}
The \LaTeX\ style defines a printed ruler which should be present in the
version submitted for review.  The ruler is provided in order that
reviewers may comment on particular lines in the paper without
circumlocution.  If you are preparing a document using a non-\LaTeX\
document preparation system, please arrange for an equivalent ruler to
appear on the final output pages.  The presence or absence of the ruler
should not change the appearance of any other content on the page.  The
camera ready copy should not contain a ruler. (\LaTeX\ users may uncomment
the \verb'\cvprfinalcopy' command in the document preamble.)  Reviewers:
note that the ruler measurements do not align well with lines in the paper
--- this turns out to be very difficult to do well when the paper contains
many figures and equations, and, when done, looks ugly.  Just use fractional
references (e.g.\ this line is $095.5$), although in most cases one would
expect that the approximate location will be adequate.

\subsection{Mathematics}

Please number all of your sections and displayed equations.  It is
important for readers to be able to refer to any particular equation.  Just
because you didn't refer to it in the text doesn't mean some future reader
might not need to refer to it.  It is cumbersome to have to use
circumlocutions like ``the equation second from the top of page 3 column
1''.  (Note that the ruler will not be present in the final copy, so is not
an alternative to equation numbers).  All authors will benefit from reading
Mermin's description of how to write mathematics:
\url{http://www.pamitc.org/documents/mermin.pdf}.

Finally, you may feel you need to tell the reader that more details can be
found elsewhere, and refer them to a technical report.  For conference
submissions, the paper must stand on its own, and not {\em require} the
reviewer to go to a techreport for further details.  Thus, you may say in
the body of the paper ``further details may be found
in~\cite{Authors14b}''.  Then submit the techreport as additional material.
Again, you may not assume the reviewers will read this material.

Sometimes your paper is about a problem which you tested using a tool which
is widely known to be restricted to a single institution.  For example,
let's say it's 1969, you have solved a key problem on the Apollo lander,
and you believe that the CVPR70 audience would like to hear about your
solution.  The work is a development of your celebrated 1968 paper entitled
``Zero-g frobnication: How being the only people in the world with access to
the Apollo lander source code makes us a wow at parties'', by Zeus \etal.

You can handle this paper like any other.  Don't write ``We show how to
improve our previous work [Anonymous, 1968].  This time we tested the
algorithm on a lunar lander [name of lander removed for blind review]''.
That would be silly, and would immediately identify the authors. Instead
write the following:
\begin{quotation}
\noindent
   We describe a system for zero-g frobnication.  This
   system is new because it handles the following cases:
   A, B.  Previous systems [Zeus et al. 1968] didn't
   handle case B properly.  Ours handles it by including
   a foo term in the bar integral.

   ...

   The proposed system was integrated with the Apollo
   lunar lander, and went all the way to the moon, don't
   you know.  It displayed the following behaviours
   which show how well we solved cases A and B: ...
\end{quotation}
As you can see, the above text follows standard scientific convention,
reads better than the first version, and does not explicitly name you as
the authors.  A reviewer might think it likely that the new paper was
written by Zeus \etal, but cannot make any decision based on that guess.
He or she would have to be sure that no other authors could have been
contracted to solve problem B.
\medskip

\noindent
FAQ\medskip\\
{\bf Q:} Are acknowledgements OK?\\
{\bf A:} No.  Leave them for the final copy.\medskip\\
{\bf Q:} How do I cite my results reported in open challenges?
{\bf A:} To conform with the double blind review policy, you can report results of other challenge participants together with your results in your paper. For your results, however, you should not identify yourself and should not mention your participation in the challenge. Instead present your results referring to the method proposed in your paper and draw conclusions based on the experimental comparison to other results.\medskip\\

\begin{figure}[t]
\begin{center}
\fbox{\rule{0pt}{2in} \rule{0.9\linewidth}{0pt}}
   %\includegraphics[width=0.8\linewidth]{egfigure.eps}
\end{center}
   \caption{Example of caption.  It is set in Roman so that mathematics
   (always set in Roman: $B \sin A = A \sin B$) may be included without an
   ugly clash.}
\label{fig:long}
\label{fig:onecol}
\end{figure}

\subsection{Miscellaneous}

\noindent
Compare the following:\\
\begin{tabular}{ll}
 \verb'$conf_a$' &  $conf_a$ \\
 \verb'$\mathit{conf}_a$' & $\mathit{conf}_a$
\end{tabular}\\
See The \TeX book, p165.

The space after \eg, meaning ``for example'', should not be a
sentence-ending space. So \eg is correct, {\em e.g.} is not.  The provided
\verb'\eg' macro takes care of this.

When citing a multi-author paper, you may save space by using ``et alia'',
shortened to ``\etal'' (not ``{\em et.\ al.}'' as ``{\em et}'' is a complete word.)
However, use it only when there are three or more authors.  Thus, the
following is correct: ``
   Frobnication has been trendy lately.
   It was introduced by Alpher~\cite{Alpher02}, and subsequently developed by
   Alpher and Fotheringham-Smythe~\cite{Alpher03}, and Alpher \etal~\cite{Alpher04}.''

This is incorrect: ``... subsequently developed by Alpher \etal~\cite{Alpher03} ...''
because reference~\cite{Alpher03} has just two authors.  If you use the
\verb'\etal' macro provided, then you need not worry about double periods
when used at the end of a sentence as in Alpher \etal.

For this citation style, keep multiple citations in numerical (not
chronological) order, so prefer \cite{Alpher03,Alpher02,Authors14} to
\cite{Alpher02,Alpher03,Authors14}.


\begin{figure*}
\begin{center}
\fbox{\rule{0pt}{2in} \rule{.9\linewidth}{0pt}}
\end{center}
   \caption{Example of a short caption, which should be centered.}
\label{fig:short}
\end{figure*}

%------------------------------------------------------------------------
\subsection{Formatting your paper}

All text must be in a two-column format. The total allowable width of the
text area is $6\frac78$ inches (17.5 cm) wide by $8\frac78$ inches (22.54
cm) high. Columns are to be $3\frac14$ inches (8.25 cm) wide, with a
$\frac{5}{16}$ inch (0.8 cm) space between them. The main title (on the
first page) should begin 1.0 inch (2.54 cm) from the top edge of the
page. The second and following pages should begin 1.0 inch (2.54 cm) from
the top edge. On all pages, the bottom margin should be 1-1/8 inches (2.86
cm) from the bottom edge of the page for $8.5 \times 11$-inch paper; for A4
paper, approximately 1-5/8 inches (4.13 cm) from the bottom edge of the
page.

%-------------------------------------------------------------------------
\subsection{Margins and page numbering}

All printed material, including text, illustrations, and charts, must be kept
within a print area 6-7/8 inches (17.5 cm) wide by 8-7/8 inches (22.54 cm)
high.



%-------------------------------------------------------------------------
\subsection{Type-style and fonts}

Wherever Times is specified, Times Roman may also be used. If neither is
available on your word processor, please use the font closest in
appearance to Times to which you have access.

MAIN TITLE. Center the title 1-3/8 inches (3.49 cm) from the top edge of
the first page. The title should be in Times 14-point, boldface type.
Capitalize the first letter of nouns, pronouns, verbs, adjectives, and
adverbs; do not capitalize articles, coordinate conjunctions, or
prepositions (unless the title begins with such a word). Leave two blank
lines after the title.

AUTHOR NAME(s) and AFFILIATION(s) are to be centered beneath the title
and printed in Times 12-point, non-boldface type. This information is to
be followed by two blank lines.

The ABSTRACT and MAIN TEXT are to be in a two-column format.

MAIN TEXT. Type main text in 10-point Times, single-spaced. Do NOT use
double-spacing. All paragraphs should be indented 1 pica (approx. 1/6
inch or 0.422 cm). Make sure your text is fully justified---that is,
flush left and flush right. Please do not place any additional blank
lines between paragraphs.

Figure and table captions should be 9-point Roman type as in
Figures~\ref{fig:onecol} and~\ref{fig:short}.  Short captions should be centred.

\noindent Callouts should be 9-point Helvetica, non-boldface type.
Initially capitalize only the first word of section titles and first-,
second-, and third-order headings.

FIRST-ORDER HEADINGS. (For example, {\large \bf 1. Introduction})
should be Times 12-point boldface, initially capitalized, flush left,
with one blank line before, and one blank line after.

SECOND-ORDER HEADINGS. (For example, { \bf 1.1. Database elements})
should be Times 11-point boldface, initially capitalized, flush left,
with one blank line before, and one after. If you require a third-order
heading (we discourage it), use 10-point Times, boldface, initially
capitalized, flush left, preceded by one blank line, followed by a period
and your text on the same line.

%-------------------------------------------------------------------------
\subsection{Footnotes}

Please use footnotes\footnote {This is what a footnote looks like.  It
often distracts the reader from the main flow of the argument.} sparingly.
Indeed, try to avoid footnotes altogether and include necessary peripheral
observations in
the text (within parentheses, if you prefer, as in this sentence).  If you
wish to use a footnote, place it at the bottom of the column on the page on
which it is referenced. Use Times 8-point type, single-spaced.


%-------------------------------------------------------------------------
\subsection{References}

List and number all bibliographical references in 9-point Times,
single-spaced, at the end of your paper. When referenced in the text,
enclose the citation number in square brackets, for
example~\cite{Authors14}.  Where appropriate, include the name(s) of
editors of referenced books.

\begin{table}
\begin{center}
\begin{tabular}{|l|c|}
\hline
Method & Frobnability \\
\hline\hline
Theirs & Frumpy \\
Yours & Frobbly \\
Ours & Makes one's heart Frob\\
\hline
\end{tabular}
\end{center}
\caption{Results.   Ours is better.}
\end{table}

%-------------------------------------------------------------------------
\subsection{Illustrations, graphs, and photographs}

All graphics should be centered.  Please ensure that any point you wish to
make is resolvable in a printed copy of the paper.  Resize fonts in figures
to match the font in the body text, and choose line widths which render
effectively in print.  Many readers (and reviewers), even of an electronic
copy, will choose to print your paper in order to read it.  You cannot
insist that they do otherwise, and therefore must not assume that they can
zoom in to see tiny details on a graphic.

When placing figures in \LaTeX, it's almost always best to use
\verb+\includegraphics+, and to specify the  figure width as a multiple of
the line width as in the example below
{\small\begin{verbatim}
   \usepackage[dvips]{graphicx} ...
   \includegraphics[width=0.8\linewidth]
                   {myfile.eps}
\end{verbatim}
}


%-------------------------------------------------------------------------
\subsection{Color}

Please refer to the author guidelines on the CVPR 2020 web page for a discussion
of the use of color in your document.

%------------------------------------------------------------------------

%-------------------------------------------------------------------------

\end{document}
